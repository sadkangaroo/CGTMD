\documentclass[12pt]{cgtmd}
\begin{document}
\handout{1}{July 4, 2013}{Problem Set 1}

\begin{enumerate}
    \item 
        \begin{enumerate}
            \item Only one. 

                We will prove that the only NE is when each player take Rock, Paper and Scissors with equal probility. For convenience we use a triple to represent the probality of Rock, Paper and Scissors.

                First we prove this is a NE. Notice for each player, given the other's strategy $(\frac{1}{3}, \frac{1}{3}, \frac{1}{3})$, Let this players strategy be $(p_1, p_2, p_3)$, then $u = \sum(\frac{1}{3}p_i - \frac{1}{3}p_i) = 0$, so any mixed strategy is a best response. Particularly, $(\frac{1}{3}, \frac{1}{3}, \frac{1}{3})$ is a best mixed response.

                On the other hand, if in another NE one player's mixed strategy $(p_1, p_2, p_3)$, $p_i = \frac{1}{3}$ does not hold. Let's just suppose $p_1$ is the biggest one. Then the best mixed response for the other will be $(0, 1, 0)$, then $(p_1, p_2, p_3)$ should be $(0, 0, 1)$, that is a controdiction.
            \item We use a pair to represent the probility of B and S. 

                $u(B, B) = (2, 1), u(S, S) = (1, 2)$, others euqal to zero.

                Let $\sigma_1 = (\frac{2}{3}, \frac{1}{3}), \sigma_2 = (\frac{1}{3}, \frac{2}{3})$, we will prove this is a mixed NE. On one hand, iven $\sigma_2 = (\frac{1}{3}, \frac{2}{3})$. Let $\sigma_1$ be $(p, q)$. Then $u_1 = 2\frac{1}{3}p + \frac{2}{3}q = \frac{2}{3}(p + q) = \frac{2}{3}$, so  the strategy of player 1 is a best mixed response. By symmetry, this is true for player 2 either.
            \item There are two pure equilibrum in BoS. Both B and both S. Since in each case, for each player, change to another strategy won't get more score.
            \item In the constructed game, each player can choose a real number, and his utility is equal to the number he choose. Obviously, no matter what strategy one player choose, he can always change to a better one.
        \end{enumerate}
    \item
        \begin{enumerate}
            \item Let's just suppose $n > 1$.
                First we formulate it as a normal form game.

                \begin{align*}
                    N &= \{1, 2, 3, \cdots, n\} \\
                    S &= N_+^n \\
                    u_i &= \left\{ \begin{array}{ll}
                        v_i - s_i  & s_i = s_{max} \text{ and } i < j \text{ if } s_j = s_{max}\\
                        0 & \text{otherwise}
                    \end{array} \right.
                \end{align*}

                Then, we will prove, if there is a NE, then player 1 obtains the object. If not, then Let $k$ be the one obtains the object, $k \neq 1$. Then we have $s_k \leq v_k$, since any player won't choose to lose money. Thus $s_1 < s_k \leq v_k < v_1$. Then player 1 can change his bid to $v_k$ to win this auction and win non-zero profit.

                Afterwards, let's find all the NE. We have proved in a NE, player 1 submits the highest bid. If there are no other player submit a bid as high as player 1, player 1 can change his bid lower. In addition, if player 1 submit a bid lower than $v_2 - 1$, player2 can change his bid to $v_2 - 1$. So far, if a strategy profile is a NE, it should has three properties as following   
                \begin{itemize}
                    \item $v_1 \geq s_1 \geq v_2 - 1$
                    \item $\forall j, s_j \leq s_1$
                    \item $\exists j, s_j = s_i$
                \end{itemize}
                And we can easily verify that if the three properties are satisfied, it will be a NE.
            \item First we formalize weak dominance.

                A strategy $s_i$ weakly dominate $s_i'$ if
                $$\forall s_{-i}, u_i(s_i, s_{-i}) \geq u_i(s_i', s_{-i})$$
                and a strategy is weakly dominant if it weakly dominate any other strategies.

                Then we will prove for any player $i$, bid for $v_i$ is a weak dominance. In the first case, player $i$ wins the auction. If he wants to get more profit, he will pay less than he should pay, and thus he cannot win the auction, that is a controdiction. In the second case, player $i$ doesn't win the auction, then obviously he cannot get more profit.

                Finally, let's consider a equilibriam in which the winner is not player 1. 
                $$v_1 = 3, v_2 = 2, v_1 = 1$$
                $$s_1 = 1, s_2 = 10, s_3 = 2$$
                It is indeed a NE and in this case player 2 wins the auction.

        \end{enumerate}
    \item First we show the sequence as below.
        \begin{align*}
            S^1 &= \{1, 2, 3, \cdots, 100\}^{15}\\
            S^2 &= \{1, 2, 3, \cdots, 99\}^{15}\\
            S^3 &= \{1, 2, 3, \cdots, 98\}^{15}\\
            \vdots\\
            S^{100} &= \{1\}^{15}
        \end{align*}
        Then we will explain why the sequence is like that. 

        Here we use a induction. We denote $1/3$ of the class average as $X$. Suppose our last strategy profile is $\{1, 2, 3, \cdots, k\}^{15}$, then for each player, strategy $k$ is strictly dominated by $1$, since $X \leq k / 3 < (1 + k) / 2$, which means strategy $1$ is always better than strategy $k$. 
\end{enumerate}

\end{document}








