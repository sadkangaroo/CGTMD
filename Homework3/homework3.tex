\documentclass[12pt]{cgtmd}

\begin{document}
\handout{3}{July 12, 2013}{Problem Set 3}

\begin{enumerate}
    \item 
        \begin{enumerate}
            \item Suppose there are $k$ players choosing route two and $(n - k)$ players choosing route one. Then the total cost is
                $$f(k) = (n - k) + k(\frac{k}{n})^d$$
                Take the derivative of it, when taking the minimal value, we get 
                $$k_0 = \frac{n}{(d + 1)^{\frac{1}{d}}}$$
                We can easily check that in this case k takes its minimal value.
                On the other hand, consider the value of $k$, we find that only when $k = n$ and $k = n - 1$, it is a NE.
                Now we have $f(n) = n$ and $f(n - 1) = 1 + (n - 1)(\frac{n - 1}{n})^d$, obviously $f(n) > f(n - 1)$.
                So the PoA is 
                \begin{align}
                    \frac{f(n)}{f(k_0)} = \frac{1}{\frac{1}{(d + 1)^{\frac{d + 1}{d}}} + 1 - \frac{1}{(d + 1)^{\frac{1}{d}}}}
                \end{align}
            \item When $d$ goes to infinity, $(d + 1)^{\frac{d + 1}{d}}$ goes to infinity, $(d + 1)^{\frac{1}{d}}$ goes to 1, then the PoA goes to infinity. 
        \end{enumerate}
    \item Let $X$ denote the network's optimal solution. If there is a NE, whose cost is greater than $NX$. Then there must be a player whose cost is greater then $X$. This player can then change his strategy to any other one in the optimal solution, the new cost must be less or equal to $X$. Thus this is not a NE, which is a contradiction.
    \item First, we will find an optimal solution. For each player, his expected cost is at least $1$, so the total expected cost is at least $n$. When each player choose a different machine, the total cost can reach n. So $c_{min} = n$.

        Then we will give an upper bound of the worst case. For each player $p$, he has a fixed cost $1$, since the expected load made by himself is always $1$. Now let's fix other players' decisions, and suppose $p$ has not assigned his distribution yet. He will choose the machines which have the least expected load and can assign probability arbitrarily among them. We shoud note that at this moment the total load of all machines is $(n - 1)$, so the expected load of machines which $p$ will choose is at most $(\frac{n - 1}{n})$. Now add back the fixed $1$ load, the expected cost of $p$ after he assigns his distribution is at most $(1 + \frac{n - 1}{n})$. Thus the expected total cost is at most $n(1 + \frac{n - 1}{n}) = 2n - 1$

        Next we will show how we can reach this upper bound. If any player use any machine with probability $\frac{1}{n}$, then according to the argument above, it is obviously a NE, and the expected total cost is actually $(2n - 1)$.

        Finally, we can calculate the PoA, which is
        $$\frac{c_{max}}{c_{min}} = \frac{2n - 1}{n} = 2 - \frac{1}{n}$$
\end{enumerate}

\end{document}








